% To change the dissertation to a Master's Thesis, include a documentclass option such as [masters], [ms], [ma], etc.
% The default option is phd. Also available are [osudraft] and [twoside]. As a reminder, documentclass options are a
% comma-separated list, e.g. \documentclass[ms,osudraft]{osudissert96}
% 
\documentclass[ms]{osudissert96}

% 
% Everything between the \documentclass and the \begin{document} is called the preamble of the document. Everything
%   between the % \begin{document} and the \end{document} is called the body of the
%   document. Define any additional commands you want here (the preamble)
%   

% SB: Put all package definitions in the following file
% SB: These packages were chosen solely by me based on my requirements and preferences. 

\usepackage[T1]{fontenc}
\usepackage{url}
% \usepackage[caption=false]{subfig}
% \usepackage[numbers]{natbib}
%\usepackage{algorithmic}
\usepackage[pdftex]{graphicx}
\usepackage{multirow}
\usepackage{multicol}
\usepackage{mdwlist}
\usepackage{textcomp}
\usepackage{amsthm}
\usepackage{amsmath}
\usepackage{amssymb}
\usepackage{psfrag}
\usepackage{graphics}
% \usepackage{epsfig}
\usepackage{epstopdf}
\usepackage{listings}
\usepackage{booktabs}
\usepackage{color}
\usepackage{footnote}
\usepackage[usenames,dvipsnames]{xcolor}
\usepackage{threeparttable}
\usepackage{etoolbox}
% \usepackage[firstpage]{draftwatermark}

% SB: According to the hyperref documentation, the algorithm package should be loaded after hyperref
\usepackage[bookmarks=true,hidelinks=false,linktocpage=false]{hyperref}
\usepackage{cite}

\usepackage{algorithm}
\usepackage{algpseudocode}
\usepackage{balance}
\usepackage{cleveref}
\usepackage{rotating}
\usepackage{etoolbox}

% SB: For generating blind text
\usepackage{lipsum}

% SB: Cleveref footnote format
\crefformat{footnote}{#2\footnotemark[#1]#3}

% SB: Input toggle-able macros
% SB: This file defines all the toggleable macros.

% SB: Define a toggle to save space, hacks to reduce whitespace to help with printing
\newtoggle{compact-space}
%\toggletrue{compact-space}
\togglefalse{compact-space}

\newtoggle{extra-materials}
% \toggletrue{extra-materials}
\togglefalse{extra-materials}

% SB: Uncommenting this line disables our comments
% \renewcommand{\grumbler}[2]{}

%%% Local Variables:
%%% mode: latex
%%% TeX-master: "dissertation"
%%% End:

% SB: Proposal-wide macros file
%% SB: This is the propsal-wide macros file. Define shorthands to your own commands.

% Contribution 1
% SB: This file contains chapter1-specific macros/shorthands


%%% Local Variables:
%%% mode: latex
%%% TeX-master: "../dissertation"
%%% End:


% Contribution 2
\input{chap2/chap2-macros}

% Contribution 2
% SB: This file contains chapter3-specific macros/shorthands




% SB: I reckon these are handy defaults.

\graphicspath{ {figs/} }

\definecolor{mygreen}{rgb}{0,0.6,0}
\definecolor{mygray}{rgb}{0.5,0.5,0.5}
\definecolor{mymauve}{rgb}{0.58,0,0.82}

\newcommand\later[1]{\begin{quote}\textcolor{darkgreen}{\textbackslash \textbf{later\{}} #1 \textcolor{darkgreen}{\}}\end{quote}}
% \renewcommand{\later}[1]{}

\newcommand\notes[1]{\begin{quote}\textcolor{darkgreen}{\textbackslash \textbf{notes\{}} #1 \textcolor{darkgreen}{\}}\end{quote}}
% \renewcommand{\notes}[1]{}



\iftoggle{compact-space}{
  % \setlength{\pdfpagewidth}{8.5in}
  % \setlength{\pdfpageheight}{11in}
  \usepackage{setspace}
  \usepackage[compact]{titlesec}
  \usepackage[top=0.7in,bottom=0.9in,left=0.8in,right=0.8in]{geometry}
  \newcommand\compactspace{\setstretch{0.9}}
  \usepackage[font=small]{caption}
  \setlength{\abovecaptionskip}{5pt}
  \setlength{\belowcaptionskip}{-15pt}
  \titleformat{\chapter}[display]   
  {\normalfont\Large\bfseries}{\chaptertitlename\ \thechapter}{20pt}{\large}   
  \titlespacing*{\chapter}{0pt}{-40pt}{30pt}
}{}

% 
% It is better to break up the dissertation into multiple files (e.g.,
% one file per chapter, as well as separate files for the abstract,
% acknowledgements, and vita).  These files are brought into the
% document using \include{} statements.  There will be times, however,
% when you don't want to print the ENTIRE dissertation.  You can limit
% what will actually be printed by using the \includeonly{} statment.
% This contains a list of the files you want printed.  Any file NOT
% listed will not be printed.  However, all page numbers, references,
% etc., will be preserved as though all the files were actually
% printed. For example, the line below would result only in chapters 1
% and 3 being printed (if it were uncommented).
% 

% \includeonly{ch1.intro,ch3.implem}

% UPDATED TEXT (2010):
% In the newest format, titles should be title case everywhere.
% 
% HISTORICAL TEXT (1996):
% In the new format, the titles of each chapter should appear in
% uppercase.  In the TOC, however, they should be in lowercase.
% The command below automates this behavior.  However, you'll have to be
% careful not to include \labels within your \chapter definitions or
% there will be problems.  If you don't want this to be automated, comment
% out the \typesetChapterTitle definition below and do your chapters in
% the form:
% \chapter[MY TITLE]{My Title}
% 
% \renewcommand\typesetChapterTitle[1]{\uppercase{#1}}
\renewcommand\typesetChapterTitle[1]{#1}

\begin{document}

%
% First, declare the parts of your title page
%

\author{Swarnendu Biswas}
\title{Ohio State College of Engineering MS/PhD Dissertation Template}
\authordegrees{XX, YY}  % Degrees thus far, not including this one.
\unit{Computer Science and Engineering}

\advisorname{WW}
\member{XX}
\member{YY}
\member{ZZ}
%\member{Yet another dude}      % Normally you will have advisor + 2 members


%
% The following creates the title page
%

\maketitle

% Next, EITHER a copyright or BLANK page.
%
%   The following creates a page used to copyright your dissertation
%
%   BACKGROUND: Even without this copyright page, your dissertation will
%               carry a common-law copyright. However, if your
%               dissertation ends up seeing wide distribution, your
%               common-law copyright is at risk of being expunged.
%               Adding this copyright page prevents that from happening.
%
%               There are NO DOWNSIDES to including a copyright page as
%               your document is automatically copyright by law anyway.
%               However, this copyright page is OPTIONAL. If you get rid
%               of it, uncomment the \blankpage that follows it so that
%               there is a blank page here. The graduate school requires
%               a page here that is either blank or carries the
%               copyright.
%
%   IMPORTANT NOTE: The graduate school requires either a copyright page
%                   here or a BLANK PAGE here. If you get rid of the
%                   copyright, uncomment the \blankpage that follows it.
%                   You should NOT have BOTH uncommented.
%

% If you get rid of \disscopyright, restore the \blankpage line after it
\disscopyright
%\blankpage

% SB: Space hack for abstract
\iftoggle{compact-space}
{\compactspace
  \begin{small}
}{}

%
% Abstract goes here.
%

\begin{abstract}
  
%INTRO

\lipsum

% FIRST PART

% SECOND PART

% THIRD PART

% Evaluation

% SUMMARY


\end{abstract}

%
%  My Dedication
%
% SB: Dedication goes after the abstract
% Dedication is not needed in candidacy proposal
\dedication{\emph{Dedicated to all associated with The Ohio State University}}

\iftoggle{compact-space}
{\end{small}
}{}

%
% UPDATED TEXT (2010):
%  The graduate school does not require an external abstract. If this
%  changes, follow the old instructions below.
%
% HISTORICAL TEXT (1996):
%  Uncomment the three lines below to generate the external abstract.  Two
%  copies of this must be turned in to the graduate school.  These lines can
%  be placed pretty much anywhere, since the page numbering should be
%  independent of the rest of the thesis
%

% \begin{externalabstract}
%   
%INTRO

\lipsum

% FIRST PART

% SECOND PART

% THIRD PART

% Evaluation

% SUMMARY


% \end{externalabstract}


%
% Bring in Acknowledgement and Vita from separate files named ``ack.tex''
% and ``vita.tex''.
%

% Not sure whether we should include ack and vita in the candidacy protocol
\begin{acknowledgements}
I thank everyone who has ever had a cow. \ldots

In reality, this is the only page of the dissertation which the author has full control of. You can write anything you
want here, and no one can tell you it's wrong (except if the margins don't line up!!!!).
\end{acknowledgements}



\begin{vita}
  
\dateitem{2011-present}{PhD, \\
  Computer Science and Engineering, \\
The Ohio State University, USA.}

\dateitem{XX}{YY, \\
  Affiliation.}

\begin{publist}
  
%% UPDATE FOR 2010:
%  Grad school only wants research publications, and it only wants those
%  research pubs that are actually published. Accepted or ``to appear''
%  publications don't count. If they look closely, they'll tell you to
%  remove any publications that aren't in print. Haivng said that, they
%  probably won't look that closely unless you put a really long list
%  here. You're tempting fate if you add instructional publications
%  though.
  
\researchpubs

\pubitem{Swarnendu Biswas.
  \newblock OSU CSE PhD Candidacy Template.
\newblock In \emph{GitHub}, pages yy--zz, April 2015.
}

% \instructpubs
%
% \pubitem{B.~Simpson, ed.,
% \newblock ``Lab notes for Cow Science 101'', 1909.}

\end{publist}

\begin{fieldsstudy}
  
% The \majorfield* uses the unit specified in the \unit command used
% earlier in your document. If you want to use a different unit, use the
% second form shown here
% \majorfield* 
  
\majorfield{Computer Science and Engineering}

  %%
  %% Note:  If there were only one field of study, the following list
  %% would best be done using the following command:
%%
%%  \onestudy{Only Topic}{Only Professor}
%%

\begin{studieslist}
\studyitem{XX}{Prof.\ XX}
\studyitem{YY}{Prof.\ YY}
\studyitem{ZZ}{Prof.\ ZZ}
\end{studieslist}

\end{fieldsstudy}

\end{vita}

 % Might not need for a candidacy proposal

% SB: Space hack for toc, lof
\iftoggle{compact-space}
{\compactspace
  \begin{small}
}{}

%
% Make the Table of Contents and other good stuff
%

\tableofcontents
\listoftables
\listoffigures

\iftoggle{compact-space}{
\end{small}
}{}

% SB: Space hack for the body
\iftoggle{compact-space}
{\compactspace
  \begin{small}
}{}

%
% The following is a list of chapters.  Each is brought in from a
% separate file using the \include{} command.
%

\newpage
\chapter{Introduction}
\label{chap:introduction}

\lipsum

\section{Motivation}
\label{sec:intro:motivation}


\section{Problem Statement}
\label{sec:intro:problem}

\section{Overview and Outline}
\label{sec:intro:overview}

In the following, we briefly outline the work presented in this proposal. 


\section{Contributions}
\label{sec:intro:contributions}

     % tell them what you are going to tell them 

\newpage

\chapter{Background and Related Work}
\label{chap:background}

\lipsum
   % describe the problem statement

\newpage

\chapter{Contribution 1}
\label{chap:contrib1}

%%% Local Variables:
%%% mode: latex
%%% TeX-master: "../dissertation"
%%% End:


\chapter{Background and Related Work}
\label{chap:background}

\lipsum


\section{Design}
\label{chapthree:design}

\lipsum

%%% Local Variables:
%%% mode: latex
%%% TeX-master: "../dissertation"
%%% End:



\section{Implementation}
\label{chap1:impl}



\section{Evaluation}
\label{chap3:eval}


\section{Summary}
\label{chap3:conclusion}




    % what is your solution
\chapter{Contribution 2}
\label{chap:contrib2}

%%% Local Variables:
%%% mode: latex
%%% TeX-master: "../dissertation"
%%% End:


\chapter{Background and Related Work}
\label{chap:background}

\lipsum


\section{Design}
\label{chapthree:design}

\lipsum

%%% Local Variables:
%%% mode: latex
%%% TeX-master: "../dissertation"
%%% End:



\section{Implementation}
\label{chap1:impl}



\section{Evaluation}
\label{chap3:eval}


\section{Summary}
\label{chap3:conclusion}




\chapter{Contribution 3}
\label{chap:contrib3}

%%% Local Variables:
%%% mode: latex
%%% TeX-master: "../dissertation"
%%% End:


\chapter{Background and Related Work}
\label{chap:background}

\lipsum


\section{Design}
\label{chapthree:design}

\lipsum

%%% Local Variables:
%%% mode: latex
%%% TeX-master: "../dissertation"
%%% End:



\section{Implementation}
\label{chap1:impl}



\section{Evaluation}
\label{chap3:eval}


\section{Summary}
\label{chap3:conclusion}




\chapter{Related Work}
\label{chap:relatedwork}

In this chapter, we compare our proposed techniques with related work.

\lipsum

%%% Local Variables:
%%% mode: latex
%%% TeX-master: "dissertation"
%%% End:

\chapter{Future Work}
\label{chap:futurework}

In this chapter, we present possible extensions and improvements to our proposed techniques.

\lipsum

%%% Local Variables:
%%% mode: latex
%%% TeX-master: "dissertation"
%%% End:

\section{Summary}
\label{chap3:conclusion}

       % conclusion stuff

%
% If you have appendices in your dissertation, you will need the
% following, else keep it commented. The following appendices are in
% files called ``app1.tex'', and ``app2.tex'', and they
% look just like any chapter.
%

% \appendix
% \include{app1}
% \include{app2}

\iftoggle{compact-space}{
\end{small}
}{}

%
% The all important bibliography file at the end of your document!! Use
% the bibstyle you (your department) like in the \bibliographystyle{}
% statement and list the name of your bibliography database file in
% the \bibliography{} statement.  In this example, ``bibfile.bib'' is
% the name of the database.  See the LaTeX manual appendix B for details
% about the bibliography database and BibTeX.
%

% SB: Space hack for bibliography, can't get spacing to work
\iftoggle{compact-space}{
  \begin{footnotesize}
  \begin{spacing}{0.6}
}{}

\bibliographystyle{plain}
\bibliography{ref}

\iftoggle{compact-space}{
\end{spacing}
\end{footnotesize}
}{}

\end{document}


%%% Local Variables:
%%% mode: latex
%%% TeX-master: t
%%% End:
